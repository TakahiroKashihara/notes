\documentclass[12pt]{jsbook}
\usepackage[dvipdfmx]{graphicx}
\usepackage{amsmath,amssymb}
\usepackage{color}
\usepackage{bm}
\usepackage{ulem}
\usepackage{calc}
\usepackage{braket}
\usepackage{nccmath}
\usepackage{ascmac}
\usepackage{here}

%%%%%%%%%%%%%%%%%
\newcommand{\la}{\langle}
\newcommand{\ra}{\rangle}
\newcommand{\ca}{c_{k,a}}
\newcommand{\cb}{c_{k,b}}
\newcommand{\cad}{c_{k,a}^{\dagger}}
\newcommand{\cbd}{c_{k,b}^{\dagger}}
\newcommand{\cma}{c_{-k,a}}
\newcommand{\cmb}{c_{-k,b}}
\newcommand{\camd}{c_{-k,a}^{\dagger}}
\newcommand{\cbmd}{c_{-k,b}^{\dagger}}
\newcommand{\Ch}{\hat{C}}
\newcommand{\C}{\mathbb{C}}
\newcommand{\R}{\mathbb{R}}
\newcommand{\pa}{\partial}

\newcommand{\tr}[1]{\mathrm{Tr}\bigl[{#1}\bigr]}
\newcommand{\Tr}[1]{\mathrm{Tr}\Bigl[{#1}\Bigr]}
\newcommand{\TrL}[1]{\mathrm{Tr}\Bigl[{#1}}
\newcommand{\ReB}[1]{\mathrm{Re}\Bigl[{#1}\Bigr]}
\newcommand{\ReBL}[1]{\mathrm{Re}\Bigl[{#1}}
\newcommand{\bk}{\bm{k}}
\newcommand{\ik}{\int \frac{d\bm{k}}{(2\pi)^d}}
\newcommand{\iw}{\int_{-\infty}^{\infty} \frac{d\omega}{2\pi}}
\newcommand{\trg}{\int \mathrm{Tr}(g)dk_xdk_y}
\newcommand{\detg}{\int \sqrt{\mathrm{det}(g)}dk_xdk_y}

\newcommand{\Ci}{\hat{C}^{-1}}
%%%%%%%%%%%%%%%%%%%
\fboxsep=0pt 
\fboxrule=1pt
\bibliographystyle{junsrt}
\begin{document}
\title{強相関電子系における波数空間上の量子計量とチャーン数の関係}

\author{柏原 昂紘 }

\maketitle
\frontmatter

\chapter*{概要}
現代の凝縮系理論においては、物理学の幾何学的側面が重要な役割を果たす。量子計量もその幾何学的量の一つで、パラメータ空間上に距離を定義する。相互作用の無い系においては、運動量空間上の量子計量とトポロジカル不変量との関係が指摘されている。本研究では、この量子計量を相互作用がある場合に拡張する(Generalized Quantum Metric、以下GQM)。このGQMは1粒子グリーンで記述される光学伝導度(つまり、光学伝導度を計算する際にバーテックス補正を無視したもの)に基づいている。我々は、自由な系においてGQMが既存の量子計量に一致することと、GQMが半正定値性を持つことを解析的に示した。したがって、GQMは既存の量子計量の拡張になっており、数学的にも計量としての性質を満たしている。さらに、チャーン絶縁体の模型として知られるQi-Wu-Zhang(QWZ)模型に斥力相互作用を加えた模型に対してGQMを数値的に計算し、GQMが相互作用がある場合にもトポロジカル不変量と関係があることを示した。GQMは一般の系でトポロジーを判定するのに役立つと期待される。

\tableofcontents

\mainmatter


\chapter{イントロダクション}
物理学の幾何学的側面は、ゲージ理論、トポロジカル欠陥の分類など\cite{nakahara2018geometry}様々な物理の理論を構築するために重要な役割を果たしてきた。近年の凝縮系物理においては、運動量空間上の固有ベクトルの幾何学的側面が大きな注目を集めている~\cite{bernevig2013topological,PhysRevX.9.041015,doi:10.1126/sciadv.1501524,Tokura2018,PhysRevLett.115.216806}。運動量空間上の幾何学的量で最も重要な量の一つはベリー曲率である。ベリー曲率の運動量空間上の積分値は量子化し、チャーン数と呼ばれる、量子ホール効果に寄与するトポロジカル不変量に比例する~\cite{PhysRevLett.49.405}。このベリー曲率と対をなすの量子計量である。量子幾何テンソルの虚部がベリー曲率に比例するのに対し、量子幾何テンソルの実部が量子計量である\cite{shapere1989geometric}。量子計量はいわゆる「計量」であり、運動量空間上に距離を定義する。近年の研究では、 超伝導~\cite{Peotta2015,PhysRevLett.126.027002,PhysRevLett.128.087002,2022arXiv220900007H}、 光学応答~\cite{PhysRevB.94.134423,PhysRevB.104.134312,PhysRevResearch.4.013217}、非エルミート物理学~\cite{PhysRevLett.127.107402,PhysRevB.103.125302}など様々な凝縮系物理学の分野において、量子計量が重要な役割を果たすことが明らかにされている。

量子計量の積分値は量子化しないものの、量子計量とトポロジーとの関係が報告されている。例えば、量子計量の積分値(具体的には、量子計量のトレースと、量子計量によって定義されるパラメータ空間の体積(量子体積))とチャーン数の関係が指摘されている~\cite{Peotta2015,PhysRevB.104.045103, PhysRevB.103.L241102,Zhang_2022}。文献~\cite{PhysRevB.104.045103}では、量子体積によってチャーン数がある程度推定できることが数値的に確かめられている。さらに、脆弱なトポロジー(fragile topology)との関係~\cite{PhysRevLett.124.167002,PhysRevLett.126.027002}やSu-Schriffer-Heeger(SSH)模型の巻き付き数との関係(文献~\cite{PhysRevLett.124.167002}のサプリを参照)も指摘されている。

物理系のトポロジーは、相互作用がある系においても盛んに研究されている~\cite{Rachel_2018,Hohenadler_2013}。本来、トポロジカル不変量は自由な系においてブロッホハミルトニアンから定義されるものであるので、相関がある場合には同様の議論ができない。しかし、トポロジカル不変量を1粒子グリーン関数で記述することによって、一般的にトポロジカル不変量を定義することが出来る\cite{PhysRevLett.105.256803,PhysRevX.2.031008,article,PhysRevB.92.085126}。例えば、上で述べたチャーン数は、グリーン関数で表される式を運動量空間と周波数空間で積分した形で書くことが出来る~\cite{PhysRevLett.105.256803,PhysRevB.83.085426}。チャーン数は元々ベリー曲率を運動量空間で積分したものだったが、相互作用がある場合にはグリーン関数の周波数依存性を消す必要があるため、周波数空間でも積分しなければならない。相互作用の効果はグリーン関数の自己エネルギーに含まれており、この一般化されたチャーン数も量子化する。

一方で、運動量空間上の量子計量の強相関系における定義は、文献~\cite{10.21468/SciPostPhysCore.5.3.040}で提案されているものの、その定義は自由な系の場合と同様にブロッホハミルトニアンの固有状態を用いており、系の情報を十分に反映したものとは言えない。(ひねり境界条件によって定義される量子計量は相関がある場合でも定義できる~\cite{Resta2011, PhysRevB.62.1666, Raffaele_Resta_2002}。しかし、その定義は系の基底状態を用いているため、一般の系で実際に計算するのは
困難である)

本研究では、1粒子グリーン関数を用いて運動量空間上の一般化された量子計量(Generalized Quantum Metric、以下GQM)を定義することを提案する。のちに示すように、GQMは自由な系では既存の定義~\cite{PhysRevB.56.12847}と一致する。また、相互作用がある場合も半正定値性を保つので、一般の系で計量として機能する。

本論文は以下のように構成される;まず、第2章では量子計量の導入を行い、量子計量と関連する量である量子幾何テンソルとベリー曲率も含めて、よく知られた基本的性質を振り返る。第3章では、電気伝導度とワニエ関数位置のゆらぎに話題を絞って、量子計量と物理量の関係を述べる。この章では、量子計量とチャーン数の関係についても解説する。第4章では、本研究の主題である、GQMの定義を述べ、自由な系において既存の定義に一致することと半正定値性を持つことを解析的に示す。第5章では、本研究の数値計算で扱った模型と理論的手法について説明する。トポロジーを調べる際は対称性が重要な役割を果たすので、模型の対称性に特について詳しく述べる。第6章では相互作用がある系でのGQMの計算結果を示す。この結果から、GQMは相互作用がある場合でもトポロジーとの関係があることが分かる。最後に、第7章で本研究の内容をまとめ、結論を述べる。

\chapter{量子計量の定義と基本的性質}
この章では、本研究の主題である量子計量を定義し、その基本的性質について説明する。あらかじめ注意しておくと、量子計量の定義には一つの固有状態に着目する場合と複数の固有状態に着目する場合との2通りがあり、本研究で強相関系に拡張されるのは複数の固有状態に着目したものである。以下では、ハミルトニアンの$n$番目の固有状態についての量子計量は$g_{ij}^{n}(\lambda)$などのように上付き添え字を示すことで区別する。

\section{一つの固有状態に対する量子計量}
量子計量や量子幾何テンソル、ベリー曲率を定義する際には、何かしらのパラメータ族$\{\lambda_i\}_i$について滑らかなハミルトニアン$H(\lambda)=H(\lambda_1,\lambda_2,\ldots)$と、滑らかな固有状態$|\Psi_n(\lambda)\ra$が必要である\cite{Resta2011}。これらを用いて、まず量子幾何テンソルが以下の式で定義される;

\begin{eqnarray}
\label{def_single_QGT}
\chi^{n}_{ij}(\lambda)=\la \pa_{\lambda_i} \Psi_n(\lambda)|(1-|\Psi_n(\lambda)\ra\la\Psi_n(\lambda)|)|\pa_{\lambda_j}\Psi_n(\lambda)\ra 
\end{eqnarray}

ただし、$|\pa_{\lambda_i}\Psi_n(\lambda)\ra $は$|\Psi_n(\lambda)\ra$の$\lambda_i$についての偏微分を表す。$(1-|\Psi_n(\lambda)\ra\la\Psi_n(\lambda)|)$は、$|\Psi_n(\lambda)\ra$の張る空間を除いた空間への射影演算子、つまり直交射影を行う演算子である。この直交射影を$Q_n$と書き、量子力学的状態間の内積を$(,)$で表すと、
\begin{eqnarray}
\chi^{n}_{ij}(\lambda)=(|\pa_{\lambda_i}\Psi_n(\lambda)\ra,Q_n|\pa_{\lambda_j}\Psi_n(\lambda)\ra)
\end{eqnarray}
となるが、ヒルベルト空間上の射影演算子の一般的性質である$Q_n^2=Q_n$およびエルミート性を用いると、
\begin{eqnarray}
\chi^{n}_{ij}(\lambda)&=(|\pa_{\lambda_i}\Psi_n(\lambda)\ra,Q_n|\pa_{\lambda_j}\Psi_n(\lambda)\ra)\\
&=(|\pa_{\lambda_i}\Psi_n(\lambda)\ra,Q_n^2|\pa_{\lambda_j}\Psi_n(\lambda)\ra)\\
&=(Q_n|\pa_{\lambda_i}\Psi_n(\lambda)\ra,Q_n|\pa_{\lambda_j}\Psi_n(\lambda)\ra)
\end{eqnarray}
と書くこともできる。つまり、$\chi^{n}_{ij}(\lambda)$は固有ベクトルを微分して射影したもの同士で内積を取ったものとわかる。これにより、量子計量と複素射影空間のFubini-Study計量\cite{nakahara2018geometry}とのつながりが分かる。

また、各$\lambda$に対し、$\{\chi^{n}_{ij}(\lambda)\}_{ij}$を行列とみなすこともできる。すると、
\begin{eqnarray}
\chi^{n*}_{ij}(\lambda)&=&(Q_n|\pa_{\lambda_i}\Psi_n(\lambda)\ra,Q_n|\pa_{\lambda_j}\Psi_n(\lambda)\ra)^*\\
&=&(Q_n|\pa_{\lambda_j}\Psi_n(\lambda)\ra,Q_n|\pa_{\lambda_i}\Psi_n(\lambda)\ra)\\
&=&\chi^{n}_{ji}(\lambda)
\end{eqnarray}
であるから、$\{\chi^{n}_{ij}(\lambda)\}_{ij}$はエルミート行列、特に固有値は実である。さらに、任意の複素数列$\{z_i\}_i$に対し、
\begin{eqnarray}
\displaystyle \sum_{ij} z_i^{*}\chi^{n}_{ij}(\lambda)z_j=\displaystyle \sum_{ij} (z_i|\pa_{\lambda_i}\Psi_n(\lambda)\ra,Q_n z_j|\pa_{\lambda_j}\Psi_n(\lambda)\ra)
\end{eqnarray}
であるが、
\begin{eqnarray}
|\phi_n(\lambda)\ra = \displaystyle \sum_{i} z_i|\pa_{\lambda_i}\Psi_n(\lambda)\ra
\end{eqnarray}
とおくと、
\begin{eqnarray}
\label{SPDsingle}
\displaystyle \sum_{ij} z_i^{*}\chi^{n}_{ij}(\lambda)z_j=(|\phi_n(\lambda)\ra,Q_n|\phi_n(\lambda)\ra)\geq 0
\end{eqnarray}
が成り立つ。最後の不等号は射影演算子の半正定値性から従う。よって、$\{\chi^{n}_{ij}(\lambda)\}_{ij}$は半正定値エルミート行列である。

量子計量とベリー曲率はそれぞれ、量子幾何テンソルの実部と虚部に対応している;
\begin{eqnarray}
    g^{n}_{ij}(\lambda)&=&\mathrm{Re} \chi^{n}_{ij}(\lambda)\\
    \Omega^n_{ij}(\lambda)&=&-2\mathrm{Im}\chi^{n}_{ij}(\lambda)
\end{eqnarray}
が$\{\chi^{n}_{ij}(\lambda)\}_{ij}$がエルミートであることから、$\{g^{n}_{ij}(\lambda)\}_{ij}$は実対称行列、$\{\Omega^{n}_{ij}(\lambda)\}_{ij}$は実反対称行列となっていることが分かる。また、$\{g^{n}_{ij}(\lambda)\}_{ij}$は実対称行列としての半正定値性を持っている。この証明は量子幾何テンソルの半正定値性の証明と全く同様であるが、後にGQMの半正定値性を示すときに同じアイデアを用いるので念のため確認しておく。

任意の実数列$\{c_i\}_i$に対し、
\begin{eqnarray}
    \displaystyle \sum_{ij}c_i g^{n}_{ij}(\lambda)c_j &=\frac{1}{2}\displaystyle \sum_{ij} c_i \Big[(|\pa_{\lambda_i}\Psi_n(\lambda)\ra,Q_n|\pa_{\lambda_j}\Psi_n(\lambda)\ra)+\\
    &(|\pa_{\lambda_j}\Psi_n(\lambda)\ra,Q_n|\pa_{\lambda_i}\Psi_n(\lambda)\ra)]c_j
\end{eqnarray}
であるが、
\begin{eqnarray}
|\phi_n(\lambda)\ra = \displaystyle \sum_{i} c_i|\pa_{\lambda_i}\Psi_n(\lambda)\ra
\end{eqnarray}
とおくと、$c_i$が実数であることにより
\begin{eqnarray}
    \displaystyle \sum_{ij}c_i g^{n}_{ij}(\lambda)c_j &=(|\phi_n(\lambda)\ra,Q_n|\phi_n(\lambda)\ra)\geq 0
\end{eqnarray}
が成り立つ。


量子幾何テンソルの半正定値性は、パラメータが2つのとき(2次元系など)に量子計量とベリー曲率との間に成り立つ不等式を導くのに役立つ\cite{Peotta2015}。例えば、式(\ref{SPDsingle})において、特に$z_x=1,z_y=-i$の場合を考えれば、
\begin{eqnarray}
    g^n_{xx}(\lambda) + g^n_{yy}(\lambda) + \Omega^n_{xy}(\lambda)\geq 0
\end{eqnarray}
また、$z_x=1,z_y=-i$の場合を考えれば、
\begin{eqnarray}
    g^n_{xx}(\lambda) + g^n_{yy}(\lambda) - \Omega^n_{xy}(\lambda)\geq 0
\end{eqnarray}
が導かれる。従って、
\begin{eqnarray}
\label{Trgneq}
    \mathrm{Tr}(g^n(\lambda)) = g_{xx}^n(\lambda) + g_{yy}^n(\lambda)\geq |\Omega^n_{xy}(\lambda)| 
\end{eqnarray}
が成り立つ。ベリー曲率はチャーン数と関係する量なので、$\mathrm{Tr}(g^n(\lambda))$はチャーン絶縁体において大きくなることが期待される。この点については次章以降でも議論する。

Fubini-Study計量との関係も簡単に説明しておく。(Fubini-Study計量自体の説明は文献\cite{nakahara2018geometry}参照)例として、Su-Schriffer-Heeger(SSH) 模型のブロッホハミルトニアン\cite{asboth2016short}
\begin{eqnarray}
 H(k) = (t_1 + t_2 \cos(k))\sigma_x + t_2 \sin (k)\sigma_y
 \end{eqnarray}
 を考える。ここで、$t_1,t_2$は2種類のホッピングであり、$\sigma_x,\sigma_y$はパウリ行列である。結晶運動量$k$をパラメータ$\lambda$だと思えば、このハミルトニアンに対して量子計量が定義できる(今回の場合1次元の模型なので、量子幾何テンソルと量子計量は一致し、ベリー曲率は0)。以下では$t_1=0,t_2=1$とする。
 このとき、エネルギー固有値は$\pm 1$であるが、エネルギー$-1$に対応する固有状態は
 \begin{eqnarray}
     |u(k)\ra = (1,-\exp(ik))^{T}/\sqrt{2}
 \end{eqnarray}
 である。上で述べた定義に基づいて計算すると、
\begin{eqnarray}
\label{bydef}
g_{11}(k)=\la \pa_{k} u(k)|(1-|u(k)\ra\la u(k)|)|\pa_{k}u(k)\ra =1/4
\end{eqnarray}
と分かるが、これをFubini-Study計量から計算してみよう。固有ベクトル$|u(k)\ra$は、複素射影空間(今回は$\mathbb{C}P^1$)の元としては$[|u(k)\ra]=-\exp(ik)$であり、これを微分すると$\pa _k[|u(k)\ra]=-i\exp(ik)$となる。$\mathbb{C}P^1$の局所座標を$\xi$で表すと、Fubini-Stdudy計量はケーラー形式が
\begin{eqnarray}
\Omega = i \frac{1}{(1+|\xi|^2)^2}d\xi \wedge d \bar{\xi}
\end{eqnarray}
で与えられるようなケーラー計量である。つまり、$J$を概複素構造として、$Z$を$\mathbb{C}P^1$の接空間の元としたとき、$g(Z,Z)=\Omega(Z,JZ)$である。今回の場合は、
$ \xi = -i\exp(ik),Z = -i\exp(ik)\frac{\partial}{\partial \xi},JZ = \exp(ik)\frac{\partial}{\partial \bar{\xi}} $とすれば良い。ただし、$JZ$の係数は内積を取る際に複素共役にする必要がある。そうすれば、
\begin{eqnarray}
    g_{11}(k) &=&\Omega(Z,JZ) \\
    &=& i\cdot \frac{1}{(1+|-i\exp(ik)|^2)^2}\cdot (-i\exp(ik))\cdot(\exp(-ik))\\
    &=&1/4
\end{eqnarray}
となり、式($\ref{bydef})$の結果と一致する。

\section{複数の固有状態に対する量子計量}
前節では一つの固有状態に対する量子計量、量子幾何テンソル、ベリー曲率を定義したが、特にパラメータ$\lambda$が結晶運動量$k$でフェルミオン系である場合には、占有されている状態全てに対してこれらの量を定義されることも多い。具体的には、量子幾何テンソルが
\begin{eqnarray}
\chi_{ij}(k)=\displaystyle \sum_{n:\mathrm{occupied}}\la \pa_{k_i} u_n(k)|\Big(1-\displaystyle\sum_{m:\mathrm{occupied}}|u_m(k)\ra\la u_m(k)|\Big)|\pa_{k_j} u_n(k)\ra
\end{eqnarray}
で与えられ、量子計量とベリー曲率はこれまでと同様に
\begin{eqnarray}
    g_{ij}(k)&=&\mathrm{Re} \chi_{ij}(k)\\
    \Omega_{ij}(k)&=&-2\mathrm{Im}\chi_{ij}(k)
\end{eqnarray}
で定義される。ただし、$|u_n(k)\ra$はブロッホハミルトニアンの固有状態である。本研究において強相関系に定義を拡張するのは、この定義の量子計量であるので注意されたい。前節の場合と違うのは占有されている状態で和を取っているところであり、射影演算子
\begin{eqnarray}
    Q=1-\displaystyle\sum_{m:\mathrm{occupied}}|u_m(k)\ra\la u_m(k)|
\end{eqnarray}
は占有されていない状態への射影を行う。これに伴い、この量子計量はFubini-Study計量ではなくなっていることに注意。その代わりに、複素射影空間$\mathbb{C}P^n$ではなく複素グラスマン多様体上の計量となっている。グラスマン多様体は射影空間の一般化と呼べるもので、射影空間が直線(1次元部分空間)の集まりであるのに対し、グラスマン多様体は平面なども含めた部分空間の空間の集まりである\cite{nakahara2018geometry}。2バンドの模型で一つのバンドが占有されている系においては、この2つの定義は一致するが、一般には一致しない。また、各バンドの$\chi^n_{ij}(k)$を$n$について足し上げても$\chi_{ij}(k$)とは一般に等しくならない。

しかし、量子幾何テンソルテンソルと量子計量の半正定値性や、量子計量とベリー曲率の間の不等式は同様に成り立つ。
\chapter{量子計量と物理量の関係}
前章では述べた量子計量の定義と基本的性質について確認したが、物理量との関わりについては触れなかった。この章では、特に電気伝導度と量子計量の関係、ワニエ関数の位置揺らぎと量子計量の関係について詳しく見ていく。この章以降、$\hbar=e=k_B=1$となる単位系を用い、結晶格子の格子定数は1とする。
\section{電気伝導度との関係}
筆者が知る限り、一番よく知られているのは電気伝導度との関係である(イントロダクションで引用した論文も参照されたい)。前章での量子幾何テンソルの定義(例えば式$\ref{def_single_QGT}$)などからは、伝導度との関係は一見分かりづらいが、固有ベクトルの微分をハミルトニアンの微分に押し付けて\cite{Resta2011}
\begin{eqnarray}
\chi^{n}_{ij}(\lambda)=\displaystyle \sum_{m\neq n}\frac{\la \Psi_n(\lambda)|\pa_{\lambda_i} H(\lambda)|\Psi_m(\lambda)\ra\la\Psi_m(\lambda)|\pa_{\lambda_j} H(\lambda)|\Psi_n(\lambda)\ra}{(E_n(\lambda)-E_m(\lambda))^2}
\end{eqnarray}
と書き直せば、摂動論の公式に似ていることに気づく。したがって、摂動項が$\pa_{\lambda_j} H(\lambda)$の形で与えられるような場合には、量子幾何テンソルや量子計量、Berry曲率は物理量に現れると予想される。その例がまさに光学応答であり、ベクトルポテンシャル$A$をlength gaugeで扱う際の応答の計算では
\begin{eqnarray}
H(k+A)=H(k) + \pa_k H(k)\cdot A + \ldots
\end{eqnarray}
のようにブロッホハミルトニアンを摂動展開することになる(詳しい計算は例えば文献\cite{michishita2021effects}を参照されたい)。したがって、結晶運動量$k$をパラメータ$\lambda$とみなせば、運動量空間上の量子計量などが光学応答に現れることが分かる。実際、例えば光学伝導度との関係式\cite{PhysRevB.62.1666}
\begin{eqnarray}
\label{relation_qmk_conductivity}
\mathrm{Re} \int_0^{\infty}d\omega \frac{\sigma_{\alpha\beta}(\omega)+\sigma_{\beta\alpha}(\omega)}{2\pi\omega}=\frac{1}{(2\pi)^d}\int dk g_{\alpha\beta}(k)
\end{eqnarray}
が知られている。後の章では、この関係式を使って運動量空間上の量子計量を強相関系にも定義する。

実は、電気伝導度と関係があるのは運動量空間上の量子計量だけでなく、ひねり境界条件によって導入される量子計量も関係する。以下ではこれについて文献\cite{PhysRevB.62.1666}に基づいて解説する。

通常、物理系を理論的に扱う際には、系に周期境界条件を課すことが多い。つまり、系のサイズを$L$としたとき、波動関数は周期性
\begin{eqnarray}
\Psi(x+L)=\Psi(x)
\end{eqnarray}
を満たすと仮定される。これに対し、ひねり境界条件の下では、波動関数は周期性を持たず
\begin{eqnarray}
\Psi(x+L)=\exp(i\theta)\Psi(x)
\end{eqnarray}
のように位相因子$\theta$が伴う。この$\theta$をパラメータ$\lambda$とみなせば、ひねり角空間上の量子計量が定義できるのである。具体的には、まずゲージ変換
\begin{eqnarray}
|\Phi_{\theta}\ra = \exp(-i\theta\cdot \hat{x}/L)|\Psi\ra
\end{eqnarray}
を行う。ここで、$\hat{x}$は位置演算子である。すると、このゲージ変換と$|\Psi\ra$のひねり境界条件が打ち消しあい、$|\Phi_{\theta}\ra$は周期境界条件を満たすことが分かる。さらに、$|\Psi\ra$に対するシュレディンガー方程式を
\begin{eqnarray}
\label{twist_Seq}
    \hat{H}|\Psi\ra = E_\theta|\Psi\ra
\end{eqnarray}
とすると、$|\Phi_{\theta}\ra$は同じ$E_\theta$に対する固有値方程式
\begin{eqnarray}
\label{no_twist_Seq}
    \hat{H}_\theta|\Phi_\theta\ra = E_\theta|\Phi_\theta\ra
\end{eqnarray}
を満たす。ただし、ここで
\begin{eqnarray}
    \hat{H}_\theta=\exp(-i\theta\cdot \hat{x}/L)\hat{H}\exp(i\theta\cdot \hat{x}/L)
\end{eqnarray}
である。つまり、ひねり境界条件下でのシュレディンガー方程式($\ref{twist_Seq}$)は、周期境界条件下でのシュレディンガー方程式($\ref{no_twist_Seq}$)と等価であると分かった。さらに、いわゆるBaker–Campbell–Hausdorffの公式を使うことで$H_\theta$の具体的な表式を得ることができ、例えば強束縛模型(tight-binding model)では単にブロッホハミルトニアンを
\begin{eqnarray}
    H(k)\mapsto H(k+\theta/L)
\end{eqnarray}
と置き換えればよい。上式は、length gaugeでのベクトルポテンシャル$\theta/L$が存在する系に等価だから、結局、ひねり境界条件を課すことは、周期境界条件の下で磁束を導入することと等価であると分かった。

ひねり角空間での量子幾何テンソルは、上の$|\Phi_\theta\ra$を用いて
\begin{eqnarray}
\label{twist_QGT}
\chi_{ij}(\theta)=\la \pa_{\theta_i} \Phi_\theta|(1-|\Phi_\theta\ra\la\Phi_\theta|)|\pa_{\theta_j}\Phi_\theta\ra 
\end{eqnarray}
と定義される。量子計量とベリー曲率も同様である。この定義は相互作用がある場合にもwell-definedである点がメリットだが、現実的には相互作用がある系で多体の波動関数を求めることは非常に困難であると思われる。

ひねり境界条件はやや人工的概念で、物理量との関連が見出しにくいように思われるが、実はひねり角空間上の量子計量\cite{PhysRevB.62.1666}、ベリー曲率\cite{PhysRevB.31.3372}も伝導度との関係が指摘されている。

恐らく有名なのは後者であるが、用いるアイデアは共通である。まず、式$(\ref{twist_QGT})$を
\begin{eqnarray}
\chi_{ij}(\theta)=\displaystyle \sum_{m\neq 0}\frac{\la \Phi_{0,\theta}|\pa_{\theta_i} \hat{H}_\theta|\Phi_{m,\theta}\ra\la\Phi_{m,\theta}|\pa_{\theta_j} \hat{H}_\theta|\Phi_{0,\theta}\ra}{(E_{0,\theta}-E_{m,\theta})^2}
\end{eqnarray}
と書き直すと理解できる。ここで、添え字$0,m$はエネルギー準位を表す添え字で、$0$は基底状態を表す。伝導度の計算では、電流演算子$\pa _k H(k)$が現れるが、磁束$\theta$の下では
\begin{eqnarray}
\displaystyle \sum_k \pa _k H(k+\theta/L)c^{\dagger}_kc_k&=&\displaystyle \sum_k L \pa _\theta H(k+\theta/L)c^{\dagger}_kc_k\\
&=&L \pa _\theta\displaystyle \sum_k H(k+\theta/L)c^{\dagger}_kc_k\\
&=&L \pa _\theta\Big(\displaystyle \sum_k H(k+\theta/L)c^{\dagger}_kc_k + \hat{H}_{int}\Big)\\
&=&L \pa _\theta \hat{H}_\theta
\end{eqnarray}
となるので、電流演算子と$\pa _\theta \hat{H}_\theta$が結びつく。これにより、各$\theta$に対し、ホール伝導度が
\begin{eqnarray}
    \sigma_{xy}(\theta)\propto \Omega_{xy}(\theta)
\end{eqnarray}
と書けることが分かる。左辺はひねり角空間でのベリー曲率である。まったく同様にして、光学伝導度とひねり角での量子計量が結びつくことが分かる。

ここで、実験などで観測される伝導度はひねり角$\theta$に依存しておらず、観測されるのは$\theta$で平均を取った量に等しいと仮定する。ただし、任意の$\theta$に対して系は絶縁体とする。このとき、
\begin{eqnarray}
    \sigma_{xy} &=&\frac{1}{(2\pi)^2}\int_0^{2\pi}d\theta_x\int_0^{2\pi}d\theta_y \sigma_{xy}(\theta)\\
    &\propto& \int_0^{2\pi}d\theta_x\int_0^{2\pi}d\theta_y \quad\Omega_{xy}(\theta)
\end{eqnarray}
を得る。左辺はチャーン数なので量子化し、相互作用がある場合についてもホール伝導度が量子化することが分かった。

量子計量についても同様にして、
\begin{eqnarray}
\mathrm{Re} \int_0^{\infty}d\omega \frac{\sigma_{\alpha\beta}(\omega)+\sigma_{\beta\alpha}(\omega)}{2\pi\omega}=\frac{1}{(2\pi)^2}\int_0^{2\pi}d\theta_x\int_0^{2\pi}d\theta_y \quad g_{\alpha\beta}(\theta)
\end{eqnarray}
が導かれる。

この2つの関係式を見ると、相互作用がない場合に成り立っていた式で結晶運動量$k$を$\theta$に変えただけのものであることに気づく。これは、自由な系においてはこれら2つの量子幾何テンソルが結びついていることを示唆している。実際、自由な系においては波動関数が
\begin{eqnarray}
    |\Phi_\theta\ra = \hat{A}\displaystyle \prod_{\epsilon_n(k)<\mu} |k\ra \otimes |u_n(k+\theta/L)\ra
\end{eqnarray}
と書けることから従う。ただし、$\hat{A}$は反対称演算子(フェルミオン系)、$\mu$は化学ポテンシャルである、$|u_n(k)\ra$はブロッホハミルトニアンの固有ベクトルである。これを$\theta$について微分する際、
\begin{eqnarray}
\pa_{\theta_i}\Bigl(|k\ra \otimes |u_n(k+\theta/L)\ra\Bigr)
=|k\ra \otimes \frac{1}{L}\pa_{k_i}|u_n(k+\theta/L)\ra
\end{eqnarray}
によって$\theta$についての微分を$k$についての微分に変換できる。これにより、
\begin{eqnarray}
    \chi_{ij}(\theta) = \frac{1}{L^2}\displaystyle \sum_k \chi_{ij}(k+\theta/L)
\end{eqnarray}
と分かる。ただし、左辺の$\chi_{ij}(k)$は前章の後半で紹介した、占有された全ての状態に対しての量子幾何テンソルである。特に、2次元系においては、熱力学極限で
\begin{eqnarray}
    \chi_{ij}(\theta)=\frac{1}{(2\pi)^2}\int dk_xdk_y \chi_{ij}(k)
\end{eqnarray}
が成り立つ\cite{PhysRevB.104.045103}。この左辺は$\theta$に依存していないから、右辺も実際は$\theta$に依存していない。両辺の実部を取ることにより、量子計量についても同様の関係式が成り立つと分かる。また、虚部を見れば、ひねり角から定義されるチャーン数と運動量から定義されるチャーン数は等しいと分かる;
\begin{eqnarray}
    C=\frac{1}{2\pi}\int d\theta_x d\theta_y \Omega_{xy}(\theta)=\frac{1}{2\pi}\int dk_x dk_y \Omega_{xy}(k)
\end{eqnarray}
このことは、自由な系においてブロッホ波動関数を用いてホール伝導度を計算しても、基底状態の波動関数を用いてホール伝導度を計算しても、同じチャーン数が現れることを保証している。


\section{ワニエ関数の位置揺らぎとの関係}
ワニエ関数とは簡単に言えばブロッホ波動関数をフーリエ級数展開した時の係数であり、ブロッホ波動関数の持つ系の情報を引き継いでいる。したがって、ワニエ関数も系についての何らかの情報を担っていると考えられる。実際、ワニエ関数の対称性と局在性が系のトポロジーと関係していることが知られている\cite{PhysRevLett.121.126402,2022arXiv220900007H}

本研究では特にチャーン絶縁体を扱うので、ワニエ関数の局在性とチャーン絶縁体の関係について説明する。サイト$j$に局在したワニエ関数は
\begin{eqnarray}
    |w(j)\ra = {1 \over \sqrt{N}}\displaystyle \sum_k \exp(-ij\cdot k)|k\ra \otimes |u(k)\ra
\end{eqnarray}
で定義される\cite{asboth2016short}。ここで$N$は全粒子数である。$|u(k)\ra$にはゲージ変換の自由度$|u(k)\ra \mapsto \exp(i\alpha(k))|u(k)\ra$があるから、ワニエ関数にもゲージの自由度がある。よく知られているように、ワニエ関数の位置の平均$<x>=\la w(j)|\hat{x}|w(j)\ra$にはベリー位相が現れる\cite{PhysRevB.56.12847,asboth2016short}。ベリー位相はゲージ不変なので、$<x>$もゲージ不変である。同様にして、ワニエ関数の位置揺らぎ$<x>=<x^2>-<x>^2$も議論できるが、これはゲージ不変ではない。しかし、ゲージ不変な部分だけを取り出すことができ、それを$\Omega_I$とすると
\begin{eqnarray}
    \Omega_I = {1 \over (2\pi)^3}\int dk \quad \mathrm{Tr} (g(k)) 
\end{eqnarray}
と表されることが知られている\cite{PhysRevB.56.12847}。

チャーン絶縁体においては、ブロッホ波動関数が滑らかな解析的な形で書けない\cite{KOHMOTO1985343}ので、ワニエ関数の局在性は悪くなる。以下ではこれについてフーリエ級数の一般論から説明する。チャーン絶縁体は2次元の模型であるが、簡単のため1次元のフーリエ級数を扱う。フーリエ級数の理論については文献\cite{1980関数解析}を参考にした。

区間$I=[-\pi,\pi]$上の関数$f(\theta)$を考える。$f$は周期的、つまり$f(-\pi) = f(\pi)$を満たすとする。まず、$f$が$C^1$級であると仮定しよう。$f,f'$のフーリエ係数をそれぞれ$a_n,b_n$とすると、部分積分により
\begin{eqnarray}
    b_n &=& {1\over \sqrt{2\pi}} \int_{\pi}^{\pi} f'(\theta) \exp(-in\theta) d\theta\\
        &=& in \int_{-\pi}^{\pi} f(\theta)\exp(-in\theta)d\theta \\
        &=&ina_n
\end{eqnarray}
が成り立つ。ここで、$f'$に対するベッセルの不等式から
\begin{eqnarray}
    \displaystyle \sum_n |b_n|^2 < +\infty
\end{eqnarray}
であるので、特に$b_n\rightarrow 0\quad (n\rightarrow \infty)$。従って、
\begin{eqnarray}
    a_n = o(1/n)\quad (n\rightarrow \infty)
\end{eqnarray}
となる。同じ議論を繰り返すことにより、$f$が$C^2$級なら$a_n = o(1/n^2)\quad (n\rightarrow \infty)$、$f$が$C^3$級なら$a_n = o(1/n^3)\quad (n\rightarrow \infty)$と分かる。従って、$f$が滑らかなら$a_n$はどんな冪よりも速く$0$に収束する、すなわち指数的に収束する。$f$が滑らかで無い場合は上の議論はいつか破綻し、$a_n$は冪的に収束する。ワニエ関数に話を戻せば、$|u(k)\ra$が滑らかなら、ワニエ関数が指数的に局在し、$|u(k)\ra$に特異性があれば、ワニエ関数は指数的に局在することになる。
従って、$\Omega_I$は大きくなると考えられる。

実は、これを不等式で評価したのが式$(\ref{Trgneq})$であり、(式$(\ref{Trgneq})$は一つの固有状態に着目した場合の不等式であるが、複数の固有状態に着目した場合にも同じ不等式が成り立つ)2次元系において両辺を積分することで
\begin{eqnarray}
    \Omega_I &=& {1\over (2\pi)^2}\int dk\quad \mathrm{Tr}(g(k))\\
    &\geq& 2\pi |C|
\end{eqnarray}
を得る。ただし、$C$はチャーン数である。チャーン数が有限の場合には$\Omega_I$がある程度大きくなることが分かる。

ところで、量子計量の半正定値性から、行列としての$\{g_{ij}(k)\}_{ij}$の固有値は$0$以上である。2次元系においては$\{g_{ij}(k)\}_{ij}$は$2\times 2$行列となるので、その固有値を$\alpha(\geq 0),\beta(\geq 0)$としよう。このとき、相加平均、相乗平均の関係から
\begin{eqnarray}
    \mathrm{Tr}(g(k))&=&\alpha +\beta\\
    &\geq&2\sqrt{\alpha\beta}\\
    &=&2\sqrt{\mathrm{det}g(k)}
\end{eqnarray}
が成り立つと分かる。こうなると、$\sqrt{\mathrm{det}g(k)}$と$\Omega_{xy}(k)$の大小関係が気になるが、文献\cite{ozawa2021relations}において
\begin{eqnarray}
    2\sqrt{\mathrm{det}g(k)}\geq |\Omega_{xy}(k)| \quad (2バンドの場合は等号成立)
\end{eqnarray}
が成り立つことが示されている。両辺を積分すれば、2次元系において
\begin{eqnarray}
    \int dk \sqrt{\mathrm{det}g(k)}\geq \pi |C|
\end{eqnarray}
を得る。これらをまとめると、結局
\begin{eqnarray}
    \mathrm{Tr}(g(k))\geq 2\sqrt{\mathrm{det}g(k)}\geq |\Omega_{xy}(k)|
\end{eqnarray}
が成り立ち、各辺を積分することでチャーン数による下限が得られるということである。
\chapter{強相関電子系における運動量空間上の量子計量}
この章では、本研究の主題である、強相関電子系における運動量空間上の量子計量を1粒子グリーン関数に基づいて定義する。ただ、具体的な定義を説明する前に、相互作用がある場合に量子計量を定義する動機を説明しておく。
\section{相互作用がある系のトポロジー}
前章では量子計量とチャーン数の関係を指摘したが、トポロジカル絶縁体については相互作用がある場合にも盛んに研究されている\cite{Rachel_2018,Hohenadler_2013}。トポロジカル絶縁体の理論は、元々は自由な系について数学的にブロッホハミルトニアンを分類する理論が整備されていた\cite{2016トポロジカル絶縁体}。この理論は相互作用がないことを本質的に用いているので、相互作用がある系にこの理論を拡張するのは困難である。しかし、いくつかのトポロジカル不変量については一般の系に定義が拡張されている。拡張する際に用いるのは、ブロッホ波動関数やブロッホハミルトニアンでは無く1粒子グリーン関数である\cite{PhysRevLett.105.256803,PhysRevX.2.031008,article,PhysRevB.92.085126}。例えば、(波数空間から計算される)チャーン数については
\begin{eqnarray}
\label{Chern_correlated}
    C={\epsilon^{\alpha\beta\gamma}\over 24\pi^2}\int dw\int dk_xdk_y \mathrm{Tr}\Big(G\pa_\alpha G^{-1}G\pa_\beta G^{-1}G\pa_\gamma G^{-1}\Big)
\end{eqnarray}
と定義することで強相関系に拡張されている~\cite{PhysRevLett.105.256803,PhysRevB.83.085426}。ただし、$\epsilon^{\alpha\beta\gamma}$は完全反対称テンソルであり、$G=G(i\omega,k)$は松原グリーン関数である。また、添え字$\alpha,\beta,\gamma$は$\omega,k_x,k_y$のいずれかを指す。自由な系において左辺の周波数積分だけ実行するとベリー曲率になり、ブロッホ波動関数によって定義されるチャーン数に等しいことが分かる。

相互作用がある場合も、式$(\ref{Chern_correlated})$は量子化することが知られている~\cite{PhysRevLett.105.256803,PhysRevB.83.085426}。相互作用の効果を自己エネルギーに取り込めば、相関がある系でトポロジカル不変量が定義できたことになる。

では、同じ発想で量子計量を定義し、相互作用がある系でチャーン数と波数空間の量子計量の関係を調べることは出来ないだろうか?この疑問に答えるため、本研究では波数空間上の量子計量の一般化を試みた。

\section{グリーン関数を用いた波数空間上の量子計量(GQM)の定義}
前章では、光学伝導度と量子計量の関係式$(\ref{relation_qmk_conductivity})$に触れた。光学伝導度の計算では必ず波数積分が現れるので、その被積分関数を$\sigma_{\alpha\beta}(\omega,k)$とする。すると、式$(\ref{relation_qmk_conductivity})$は波数積分の被積分関数のレベルで等しいと分かる、すなわち
\begin{eqnarray}
    \mathrm{Re} \int_0^{\infty}d\omega \frac{\sigma_{\alpha\beta}(\omega,k)+\sigma_{\beta\alpha}(\omega,k)}{2\pi\omega}=g_{\alpha\beta}(k)
\end{eqnarray}
が成り立つと分かる。自由な系においては、$\sigma_{\alpha\beta}(\omega,k)$は1粒子グリーン関数を用いて表すことができる\cite{michishita2021effects}ので、これによって$g_{\alpha\beta}(k)$が1粒子グリーン関数で量子計量を表せたことになる(相互作用がある場合は、伝導度の計算においてバーテックス補正を無視したことになる);
\begin{eqnarray}
\label{defGQM}\tilde{g}_{\alpha\beta}(\bk)&=&\mathrm{Re}\int_0^{\infty} d\omega_1 \frac{\sigma_{\alpha\beta}(\omega_1,\bk) + \sigma_{\beta{\alpha}}(\omega_1,\bk)}{2\pi\omega_1}\\
\label{optk}\sigma_{\alpha\beta}(\omega_1,\bk)&=&-\frac{1}{\omega_1}\int \frac{d\omega}{2\pi}f(\omega)\nonumber\\
&&\times \mathrm{Tr}\Bigl[J_{\alpha\beta}G^{R-A}(\omega)\nonumber\\
&& \ \ \ \ \ \ +J_\alpha G^R(\omega+\omega_1)J_\beta G^{R-A}(\omega)\nonumber\\ 
&& \ \ \ \ \ \ +J_\alpha G^{R-A}(\omega)J_\beta G^A(\omega-\omega_1)\Bigr]
\end{eqnarray}
ただし、式(\ref{optk})の右辺では波数依存性を省略している。$G^R,G^A$はそれぞれ遅延、先進グリーン関数であり、$J_\alpha=\pa_{\alpha} H(k)$は電流演算子、$f(\omega)$はフェルミ分布関数である。また、$G^{R-A}=G^R-G^A$である。上のグリーン関数で定義された量子計量については、これまでの量子計量$g_{\alpha\beta}(k)$と区別するため$\tilde{g}_{\alpha\beta}(\bk)$と書くことにする。以下では$\tilde{g}_{\alpha\beta}(\bk)$を一般化された量子計量(Generalized Quantum Metric, GQM)と呼ぶ。グリーン関数の中に自己エネルギーを含めれば、相関がある場合にも量子計量が定義できたことになる。

GQMが自由な系において絶対零度$T=0$で既存の定義と一致することを確かめよう。その前に、グリーン関数の基本的性質について確認しておく。ただし、以下では絶対零度$T=0$のみを考える。

遅延グリーン関数、先進グリーン関数はそれぞれ
\begin{eqnarray}
    G^R_{ab}(t,k) &= -i\theta(t)\langle c_{k,a}(t)c_{k,b}^{\dagger} + c_{k,b}^{\dagger}c_{k,a}(t)\rangle\\
G^A_{ab}(t,k) &= i\theta(-t)\langle c_{k,a}(t)c_{k,b}^{\dagger} + c_{k,b}^{\dagger}c_{k,a}(t)\rangle
\end{eqnarray}
で定義される。ただし、$a,b$は内部自由度を表す。これをレーマン表示してフーリエ変換すると、
\begin{eqnarray}
    G^R_{ab}(\omega,k) &= \displaystyle \sum_n \frac{\la 0|\ca |n\ra \la n| \cbd | 0\ra}{E_0-E_n + \omega + i\eta } + \frac{\la 0|\cbd |n\ra \la n| \ca |0\ra}{E_n - E_0 + \omega + i\eta}\\
G^A_{ab}(\omega,k) &= \displaystyle \sum_n \frac{\la 0|\ca |n\ra \la n | \cbd | 0\ra}{E_0-E_n + \omega -i\eta} + \frac{\la 0| \cbd |n \ra \la n | \ca | n\ra}{E_n - E_0 + \omega - i\eta}
\end{eqnarray}
を得る。ただし、$|0\ra$は基底状態、$|n\ra$は$n$番目のエネルギー固有状態、$E_0,E_n$はそれに対応するエネルギー固有値である。また、分母に表れる$\eta$は正の微小量である。この表式から、
\begin{eqnarray}
    G^R(\omega,k) = G^{A\dagger}(\omega,k)
\end{eqnarray}
の関係があることが分かる。

さらに、スペクトル関数は

\begin{eqnarray}
    A_{ab}(\omega,k)&=& -\frac{1}{2\pi i} (G^R_{ab}(\omega,k)-G^A_{ab}(\omega,k))\\
    &=&\displaystyle \sum_n \la 0|\ca|n\ra \la n|\cbd|0\ra \delta(E_0 - E_n + \omega) \\
    &+& \la 0|\cbd|n\ra\la n| \ca | 0\ra \delta (E_n-E_0 + \omega)
\end{eqnarray}
書けるので、$A_{ab}(\omega,k)$はエルミート行列である。さらに、任意の複素数列$\{z_a\}$に対し、
\begin{eqnarray}
    \displaystyle \sum_{a,b} z^*_a A_{ab}(\omega,k)z_b &=& \displaystyle \sum_{a,b,n} \la 0|z^*_a \ca |n\ra\la n | \cbd z_b |0\ra \delta(E_0-E_n + \omega) \\
    &+&  \la 0|z_b \cbd|n\ra\la n| z_a^*\ca | 0\ra \delta (E_n-E_0 + \omega)
\end{eqnarray}
であるが、ここで
\begin{align}
|\phi_k\ra &= \displaystyle \sum_a \cad z_a |0\ra\\ 
|\psi_k\ra &= \displaystyle \sum_a \ca z_a^* |0\ra
\end{align}
を用いると、
\begin{align}
\displaystyle \sum_{a,b} z^*_aA_{ab}(\omega,k)z_b &= \displaystyle \sum_n |\la \phi_k|n\ra|^2 \delta(E_0 - E_n + \omega) + |\la \psi_k|n\ra |^2\delta(E_n-E_0 + \omega)\geq 0
\end{align}
つまり、エルミート行列$A(\omega,k)$半正定値である。これにより、$A(\omega,k)$の固有値が全て0以上であることがわかる。

話を戻して、自由な系での式$(\ref{optk})$を評価しよう。簡単のため、化学ポテンシャルは$0$とする。まず、第一項が純虚数であること、したがって式$(\ref{defGQM})$に寄与しないことを示そう。上で確認したように、$-i(G^R - G^{A})$はエルミート行列であるから、ユニタリー行列で対角化できる。対応する固有ベクトルを$|n\ra$、固有値を$A_n(\geq 0)$とすると、
\begin{align}
\mathrm{Tr}[J_{\alpha\beta}(G^{R} - G^{A})] &= \displaystyle \sum_{n,m}\la n|J_{\alpha\beta}|m\ra\la m|(G^{R}-G^{A})|n\ra\\
                                   &= \displaystyle \sum_{n,m}\la n|J_{\alpha\beta}|m\ra(-iA_n\delta_{mn})\\
\label{flast}                                   &= \displaystyle \sum_{n}-iA_n\la n |J_{\alpha\beta}|n\ra
\end{align}
である。$J_{\alpha\beta}$エルミート行列なので、左辺は純虚数だと分かる。この事実は相互作用がある場合にも変わらない。

次に、式$(\ref{optk})$の第二項を評価しよう。ここでは、ブロッホハミルトニアンの固有状態$|n\ra$を用い、対応する固有値を$E_n$とする。また、行列要素$\la n| J_\alpha|m\ra$を$(J_\alpha)_{nm}$などと書く。
\begin{eqnarray}
(第二項)
&=& \mathrm{Tr}(J_\alpha G^R(\omega+\omega_1)J_\beta (G^R-G^A))\\
         &=& (J_\alpha)_{nm}(G^R(\omega+\omega_1))_{ml}(J_\beta)_{lk}((G^R)_{ln}-(G^A)_{ln})\\
         &=& (J_\alpha)_{nm}\frac{\delta_{ml}}{\omega + \omega_1 - E_m + i\eta}(J_\beta)_{lk}(\frac{\delta_{ln}}{\omega - E_n + i\eta}-\frac{\delta_{ln}}{\omega - E_n - i\eta}))\\
         &=&\frac{-2\pi i (J_\alpha)_{nm}(J_\beta)_{mn}}{\omega + \omega_1 -E_m + i\eta}\delta(\omega - E_n)
\end{eqnarray}
したがって、
\begin{eqnarray}
\int \frac{d\omega}{2\pi}(第二項) = -i\frac{(J_\alpha)_{nm}(J_\beta)_{mn}}{\omega_1-(E_m-E_n)+i\eta}f(E_n)
\end{eqnarray}
同様にして、
\begin{eqnarray}
(第三項) = \frac{-2\pi i (J_\alpha)_{nm}(J_\beta)_{mn}}{\omega - \omega_1 -E_n - i\eta}\delta(\omega - E_m)
\end{eqnarray}
\begin{eqnarray}
\int \frac{d\omega}{2\pi}(第三項) = i\frac{(J_\alpha)_{nm}(J_\beta)_{mn}}{\omega_1-(E_m-E_n)+i\eta}f(E_m)
\end{eqnarray}
であるから、
\begin{align}
&\ReB{\sigma_{\alpha\beta}(\omega_1,\bk)}\nonumber\\
&=  \frac{1}{\omega_1}\sum_{nm}\Bigl(f(E_n)-f(E_m)\Bigr)\mathrm{Re}\Bigl[\frac{i(J_\alpha)_{nm}(J_\beta)_{mn}}{\omega_1-(E_m-E_n) + i\eta}\Bigr].
\end{align}
$\alpha,\beta$について対称化すると、
\begin{align}
&\ReB{\frac{\sigma_{\alpha\beta}(\omega_1,\bk) + \sigma_{\beta \alpha}(\omega_1,\bk)}{2}}\nonumber\\
&= \frac{1}{2\omega_1}\sum_{n\neq m}\Bigl(f(E_n)-f(E_m)\Bigr)\nonumber\\
& \ \ \ \ \times\mathrm{Re}\Bigl[\frac{i((J_\alpha)_{nm}(J_\beta)_{mn}+(J_\beta )_{nm}(J_\alpha)_{mn})}{\omega_1-(E_m-E_n) + i\eta}\Bigr]
\end{align}
ここで、$(J_\alpha)_{nm}(J_\beta)_{mn}+(J_\beta )_{nm}(J_\alpha)_{mn}\in \R$に注意すると、
\begin{align}
&\ReB{\frac{\sigma_{\alpha\beta}(\omega_1,\bk) + \sigma_{\beta \alpha}(\omega_1,\bk)}{2}}\nonumber\\
&=\frac{\pi }{\omega_1}\sum_{n\neq m} \delta(\omega_1-(E_m-E_n))\Bigl(f(E_n)-f(E_m)\Bigr)\nonumber\\
& \ \ \ \ \ \ \ \ \ \ \ \ \label{fnfm}\times\ReB{(J_\alpha)_{nm}(J_\beta)_{mn}}
\end{align}
を得る。$T=0$においては、上式は
\begin{eqnarray}
\frac{\pi}{2\omega_1}\displaystyle \sum_{n:E_n<0}\displaystyle \sum_{m:E_m>0}((J_\alpha)_{nm}(J_\beta)_{mn}+(J_\beta )_{nm}(J_\alpha)_{mn})\nonumber \\
\times [\delta(\omega_1-(E_m-E_n))-\delta(\omega_1-(E_n-E_m))]
\end{eqnarray}
と書ける。以上より、
\begin{eqnarray}
\tilde{g}_{\alpha\beta}(\bk) &=& \mathrm{Re} \;\int_{0}^{\infty}d\omega_1\frac{\sigma_{\alpha\beta}(\omega_1,\bk) + \sigma_{\beta{\alpha}}(\omega_1,\bk)}{2\pi\omega_1}\\
&=& \frac{1}{2}\displaystyle \sum_{n:E_n<0}\displaystyle \sum_{m:E_m>0} \Bigl[\frac{\langle n|\partial_\alpha H |m\rangle\langle m|\partial_\beta H |n\rangle}{(E_n-E_m)^2}\nonumber\\
&& \ \ \ \ \ \  + (\alpha \leftrightarrow \beta)\Bigr]\nonumber\\
&=& \mathrm{Re} \displaystyle \sum_{n:E_n<0}\displaystyle \sum_{m:E_m>0} \Bigl[\frac{\langle n|\partial_\alpha H |m\rangle\langle m|\partial_\beta H |n\rangle}{(E_n-E_m)^2}\\
&=&{g}_{\alpha\beta}(\bk)\label{existing}
\end{eqnarray}
が示された。
\section{GQMの半正定値性}
次に、$GQM$の半正定値性を示そう。前節ではブロッホハミルトニアンの固有状態を用いて解析したが、一般の系ではこの方法は不便である。そこで、この説ではスペクトルに対応する$i(G^{R} - G^{A})$の固有状態$|n\ra$を用いよう。固有値を$A_n$とすれば、前節で述べたように$A_n\geq 0$である。また、解析しやすくするために、GQMの式を$G^R-G^A$で表そう$(G^Rのみ、またはG^Aのみの項は扱いづらい)$。

まず、第二項は
\begin{align}
(第二項)&=\mathrm{Re} \; \mathrm{Tr}\Bigl[J_\alpha G^R(\omega+\omega_1)J_\beta (G^R-G^A) +(\alpha \leftrightarrow \beta)\Bigr]\nonumber\\ 
&=\mathrm{Re} \; \mathrm{Tr}\Bigl[J_\alpha G^R(\omega+\omega_1)J_\beta (G^R-G^A) \nonumber\\
&+ J_\beta G^R(\omega+\omega_1)J_\alpha (G^R-G^A)\Bigr]\nonumber\\
&= \mathrm{Re} \; \mathrm{Tr}\Bigl[J_\alpha G^R(\omega+\omega_1)J_\beta (G^R-G^A)\nonumber \\
&+ \{J_\beta G^R(\omega+\omega_1)J_\alpha (G^R-G^A)\}^{\dagger}\Bigr]\nonumber\\
&= \mathrm{Re} \; \mathrm{Tr}\Bigl[J_\alpha G^R(\omega+\omega_1)J_\beta (G^R-G^A)\nonumber \\
&+(G^A-G^R)J_\alpha G^A(\omega + \omega_1)J_\beta\Bigr]\nonumber\\
&= \mathrm{Re} \; \mathrm{Tr}\Bigl[J_\alpha G^R(\omega+\omega_1)J_\beta (G^R-G^A)\nonumber\\
&+J_\alpha G^A(\omega+\omega_1)J_\beta(G^A-G^R)\Bigr]\quad \nonumber\\
&= \mathrm{Re} \; \mathrm{Tr}\Bigl[J_\alpha (G^R(\omega+\omega_1)-G^A(\omega+\omega_1))J_\beta (G^R-G^A)\Bigr]\nonumber\\
&= \mathrm{Re} \Tr{J_\alpha G^{R-A}(\omega+\omega_1)J_\beta G^{R-A}(\omega)}
\end{align}
と書ける。これを$|n\ra$を用いて書き下すと、
\begin{eqnarray}
&&\displaystyle \sum_{n,m}-A_n(\omega)A_m(\omega+\omega_1)\times \Bigl(\langle n(\omega)|J_\alpha|m(\omega + \omega_1)\rangle\nonumber\\
&&\times \langle m(\omega+\omega_1)|J_\beta|n(\omega)\rangle 
+ (\alpha\leftrightarrow\beta) \Bigr)/2.
\end{eqnarray}
となる。

第三項については、変数変換
\begin{align}
 \omega-\omega_1 \rightarrow \omega,\ \ \  f(\omega)\rightarrow f(\omega + \omega_1)
\end{align}
行ってから同様にすると、第二項と合わせて
 \begin{eqnarray}
&&\displaystyle \sum_{n,m}A_n(\omega)A_m(\omega+\omega_1)\nonumber\\
&& \times \Bigl(\langle n(\omega)|J_\alpha|m(\omega + \omega_1)\rangle\langle m(\omega+\omega_1)|J_\beta|n(\omega)\rangle \nonumber\\
&&+ (\alpha\leftrightarrow\beta) \Bigr)\times \frac{f(\omega)-f(\omega+\omega_1)}{\omega_1^2}\times \frac{1}{8\pi^2}.
\end{eqnarray}
と書けることが分かる。
ここで、
\begin{align}
h(n,m,\omega,\omega_1) = A_n(\omega)A_m(\omega+\omega_1)\frac{f(\omega)-f(\omega+\omega_1)}{8\pi^2\omega_1^2}
\end{align}
を定義する。重要なのは、スペクトル関数の半正定値性とフェルミ分布関数の単調性から$h(n,m,\omega,\omega_1)\geq 0$成り立つことである。任意の実数列${c_\alpha}$に対し、
\begin{align}
\displaystyle &\sum_{\alpha,\beta}c_\alpha \tilde{g}_{\alpha\beta}c_{\beta}\nonumber\\
&= \displaystyle 2\sum_{n,m,\alpha,\beta}\int_{0}^{\infty}d\omega_1\int_{-\infty}^{\infty}d\omega \;h(n,m,\omega,\omega_1)\;\nonumber\\
& \ \ \ \times \langle n(\omega)|c_\alpha J_\alpha|m(\omega + \omega_1)\rangle \langle m(\omega+\omega_1)|c_\beta J_\beta|n(\omega)\rangle.
\end{align}
であるが、
\begin{align}
|\phi_n(\omega)\rangle = \displaystyle \sum_{\alpha} c_\alpha J_\alpha|n(\omega)\rangle ,
\end{align}
を用いると
\begin{eqnarray}
\displaystyle \sum_{\alpha,\beta}c_\alpha \tilde{g}_{\alpha\beta}c_{\beta}&=&\displaystyle 2\sum_{n,m} \int_{0}^{\infty}d\omega_1\int_{-\infty}^{\infty}d\omega h(n,m,\omega,\omega_1)\nonumber\\
&& \ \times \langle \phi_n(\omega)| m(\omega+\omega_1)\rangle\langle m(\omega+\omega_1)|\phi_n(\omega)\rangle \nonumber\\
&\geq& 0
\end{eqnarray}
となり、GQMの半正定値性が示された。
\section{GQMとチャーン数の関係}
前節で光学伝導度とひねり角空間の量子計量との関係について述べたが、その関係は相互作用がある場合にも成り立つ。これを用いると、以下に示すように2次元系におけるGQMとチャーン数間の不等式が得られる;
\begin{eqnarray}
    {1\over (2\pi)^2}\int dk_xdk_y \mathrm{Tr}(\tilde{g}(k))&\approx& \mathrm{Re}\int_0^\infty {\sigma_{xx}(\omega) + \sigma_{yy}(\omega)\over 2\pi\omega}\nonumber\\
    &=&{1\over (2\pi)^2}\int d\theta_xd\theta_y \mathrm{Tr}(\tilde{g}(\theta))\nonumber\\
    &\geq& {1\over (2\pi)^2}\int d\theta_xd\theta_y |\Omega_{xy}(\theta)|\nonumber\\
    &\geq& {1\over 2\pi}|C|
\end{eqnarray}
ここで$C$はチャーン数である。ただし、1行目でバーテックス補正を無視しているので不等式は厳密ではないが、相互作用がある場合にもGQMはチャーン絶縁体において大きくなることが期待される。
\chapter{本研究で扱う模型と理論的手法}
本研究では、前章で定義したGQMを具体的な模型に対して数値的に計算した。以下では、その際に使用した模型について説明する。
\section{模型のハミルトニアンとその対称性}%静的平均場、ハミルトニアンの対称性、グリーン関数の対称性
使用した模型のハミルトニアンは以下の形で与えられる;
\begin{eqnarray}
H &=& H_{\mathrm{QWZ}} + H_{\mathrm{int}}-\mu \displaystyle \sum_i \Bigl(n_{i,a}+n_{i,b}\Bigr)\\
H_{\mathrm{QWZ}} &=& \displaystyle \sum_{\bk} (c^{\dagger}_{\bk,a}\; c^{\dagger}_{\bk,b})\;H(\bk)\;(c_{\bk,a}\; c_{\bk,b})^{T}\\
H(\bk) &=& \sin k_x\sigma_x + \sin k_y\sigma_y \nonumber\\
&& \ \ \ + (M+\cos k_x +\cos k_y)\sigma_z\\
H_\mathrm{int} &=& U\displaystyle \sum_i n_{i,a}n_{i,b}, 
\end{eqnarray}
ただし、$a$と$b$は擬スピンの自由度を表し、 $i$ は格子点のラベルであり、$\sigma_i$ はパウリ行列である。また、$\mu$は化学ポテンシャルである。 $M$は$a$サイトと$b$サイトのポテンシャルの差~\cite{asboth2016short}(もしくは磁場)の役割を果たす。$U$は斥力相互作用の強さを表す。$H_\mathrm{QWZ}$はQWZ模型として知られる模型であり、X.-L. Qi, Y.-S. Wu, and S.-C. Zhang~\cite{PhysRevB.74.085308,asboth2016short}によって提案された正方格子上のチャーン絶縁体のtoy modelである。この模型のチャーン数は$0 < M < 2$のとき$C=1$、$-2 < M < 0$のとき$C=-1$ である。$\vert M\vert >2$のときはトポロジカルに自明である。 $M=\pm 2$ と $M=0$ は量子臨界点であり、エネルギーギャップが閉じる~\cite{asboth2016short}。系を常にhalf-filledに保つため、$\mu=U/2$とする。

まず、$U=0$の場合について模型の対称性を調べよう。いわゆるAltland-Zirnbauer(AZ)クラスの中でどのクラスに属するかを考える。$K$を複素共役演算子として
\begin{eqnarray}
\Xi = \sigma_x K\quad 
\end{eqnarray}
とすれば、$\Xi$は反ユニタリーであり
\begin{eqnarray}
\Xi H(k)\Xi^{-1} = -H(-k)
\end{eqnarray}
を満たすから、QWZ模型は粒子正孔対称性(Particle-Hole-Symmetry,PHS)を持つ。また、チャーン数が有限になりうることから時間反転対称は有しておらず、
\begin{eqnarray}
\Xi^2=1    
\end{eqnarray}
であることからAZクラスでのクラスDに属すると分かる。

結晶対称性についても調べておこう。$k_x\to-k_x, k_y\to-k_y$とすると、 
\begin{eqnarray}
H(-k) = -\sin k_x \sigma_x + -\sin k_y \sigma_y + (M+\cos k_x + \cos k_y)\sigma_z 
\end{eqnarray}となる。従って、 
\begin{eqnarray}
    \sigma_z H(k)\sigma_z = H(-k)
\end{eqnarray} 
が成り立つ。つまり、QWZ模型は2次元での空間反転対称性を持つ。

以上の議論はブロッホハミルトニアンの行列としての性質を調べているので、相互作用がある場合には同じ形式での議論が出来ない。しかし、生成消滅演算子の変換を調べることで、相互作用の有無に依らない統一的な議論が出来る\cite{PhysRevB.83.085426}。

PHSについては、自由な場合に$\Xi = \sigma_x K$だったことから示唆されるように、
\begin{eqnarray}
\hat{C}c_{k,a}\hat{C}^{-1} &=& c^{\dagger}_{-k,b}\\
\hat{C}c_{k,b}\hat{C}^{-1} &=& c^{\dagger}_{-k,a}
\end{eqnarray}
を満たすユニタリー演算子$\hat{C}$を考えればよい。$\hat{C}$は当然$H_{\mathrm{QWZ}}$と可換であるが、相互作用がある場合に追加される項とも可換であると分かる;
\begin{eqnarray}
\hat{C}\Bigl[H_{\mathrm{int}}-\mu \displaystyle \sum_i \Bigl(n_{i,a}+n_{i,b}\Bigr)\Bigr]\hat{C}^{-1}=H_{\mathrm{int}}-\mu \displaystyle \sum_i \Bigl(n_{i,a}+n_{i,b}\Bigr)
\end{eqnarray}
空間反転対称についても、自由な場合において対応する演算子が$\sigma_z$であることから分かるように、
\begin{eqnarray}
 \hat{I}c_{k,a}\hat{I}^{-1}=c_{-k,a},\quad \hat{I}c_{k,b}\hat{I}^{-1}=-c_{-k,b} 
\end{eqnarray}
 を満たすユニタリー演算子$\hat{I}$を考えればよい。$U$の値によらず、系のハミルトニアンは$\hat{I}$と可換である;
 \begin{eqnarray}
     \hat{I}H\hat{I}^{-1}=H
 \end{eqnarray}
 以上より、ハミルトニアン$H$はPHSと空間反転対称性を持つことが分かった。

 これらの対称性により、グリーン関数、スペクトル関数、自己エネルギーが制約を受ける。以下ではこれについて説明する。

レーマン表示を用いると、QWZ模型のPHSにより 

\begin{align} 
G^R_{ab}(\omega,k) &= \displaystyle \sum_n \frac{\la 0|\Ch^{-1}\Ch\ca \Ch^{-1}\Ch |n\ra \la n|\Ch^{-1}\Ch \cbd \Ch^{-1}\Ch| 0\ra}{E_0-E_n + \omega + i\eta } \\
&+ \frac{\la 0|\Ch^{-1}\Ch\cbd \Ch^{-1}\Ch|n\ra \la n|\Ch^{-1}\Ch \ca \Ch^{-1}\Ch|0\ra}{E_n - E_0 + \omega + i\eta}\\ 
&=\displaystyle \sum_n \frac{\la 0|\cbmd |n\ra \la n| \cma | 0\ra}{E_0-E_n + \omega + i\eta } \\
&+ \frac{\la 0|\cma |n\ra \la n| \cbmd |0\ra}{E_n - E_0 + \omega + i\eta}\\
&=-G^{R*}_{ba}(-\omega,-k)
\end{align} 
が成り立つことが分かる。同様にして、
\begin{align} G^R_{aa}(\omega,k) &=-G^{R*}_{bb}(-\omega,-k)\\
G^R_{bb}(\omega,k) &=-G^{R*}_{aa}(-\omega,-k)\\
G^R_{ba}(\omega,k) &=-G^{R*}_{ab}(-\omega,-k)
\end{align} 
を得る。従って、逆行列を取ることにより
\begin{align}
\Big( \omega-H(k)+U/2-\Sigma^R(\omega,k)\Big)_{aa}=-\Big(-\omega-H^{*}(-k)+U/2-\Sigma^{R*}(-\omega,-k)\Big)_{bb} 
\end{align}  
であるから、($U/2$は化学ポテンシャル、$\Sigma^R$は遅延グリーン関数の自己エネルギー)QWZ模型の$H(k)$を当てはめると、
\begin{eqnarray}
U=\Sigma^{R}_{aa}(\omega,k)+\Sigma^{R*}_{bb}(-\omega,-k)
\end{eqnarray}
この実部と虚部について分けて見ると、
\begin{eqnarray}
    U=\mathrm{Re} \Sigma^{R}_{aa}(\omega,k)+\mathrm{Re} \Sigma^{R}_{bb}(-\omega,-k)\\
    0 = \mathrm{Im}\Sigma^{R}_{aa}(\omega,k)-\mathrm{Im}\Sigma^{R}_{bb}(-\omega,-k)
\end{eqnarray}
が満たされることが分かる。
また、スペクトル関数のトレース$\mathrm{Tr}A(\omega,k) =-{1\over\pi}\mathrm{ImTr}G^R(\omega,k)$については、 
\begin{eqnarray}
    \mathrm{Tr}A(\omega,k)=\mathrm{Tr}A(-\omega,-k)
\end{eqnarray}  
が成り立つ。

空間反転対称性についても全く同様の議論を行うことで
\begin{align}
G^R(\omega,k)_{ab}&=-G^R(\omega,-k)_{ab}\\
G^R(\omega,k)_{aa}&=G^R(\omega,-k)_{aa}\\
G^R(\omega,k)_{bb}&=G^R(\omega,-k)_{bb}
\end{align}  を得る。 したがって、スペクトル関数は 
\begin{eqnarray}
    \mathrm{Tr}A(\omega,k)=\mathrm{Tr}A(\omega,-k)
\end{eqnarray} 
を満たす。PHSと合わせると、 
\begin{eqnarray}
    \mathrm{Tr}A(\omega,k)=\mathrm{Tr}A(-\omega,k)
\end{eqnarray}
も成り立つ。
\section{トポロジーの判定手法}%effective Hamiltonian, WilsonLoop
前節では相互作用がある場合のチャーン数の表現について触れたが、より簡便な方法として、$\omega=0$での松原グリーン関数にのみ着目する方法も知られている\cite{PhysRevX.2.031008,article}。松原グリーン関数
\begin{eqnarray}
     G_{ab}(i\omega,k) &= \displaystyle \sum_n \frac{\la 0|\ca |n\ra \la n| \cbd | 0\ra}{E_0-E_n + i\omega  } + \frac{\la 0|\cbd |n\ra \la n| \ca |0\ra}{E_n - E_0 + i\omega}\\
\end{eqnarray}
において$i\omega=0$とすると、$G_{ab}(i\omega=0,k)$はエルミート行列であることが分かる。したがって、その逆行列である
\begin{eqnarray}
    -G^{-1}_{ab}(i\omega,k)=H(k) + \Sigma(i\omega=0,k)
\end{eqnarray}
もエルミートである。そこで、この有効ハミルトニアン$H_{\mathrm{eff}}(k) = H(k) + \Sigma(i\omega=0,k)$に対して
、自由な系で確立された方法に基づいたトポロジカル不変量が定義できる。以上の方法は、自己エネルギーの周波数依存性について詳しく知る必要が無いため、非常に簡単なトポロジーの判定手法であり、しかも判定結果も正しいことが知られている。

本研究においては、松原グリーン関数の自己エネルギーではなく遅延グリーン関数の自己エネルギーを用いた。一般に、松原グリーン関数と遅延グリーン関数には

\begin{eqnarray}
    G_{ab}(i\omega,k) \overset{i\omega\to \omega+i\eta}{\Longrightarrow}G^{R}_{ab}(\omega,k)
\end{eqnarray}
の関係があるので、
\begin{eqnarray}
    G^{R}_{ab}(\omega=0,k)=G_{ab}(i\eta,k)
\end{eqnarray}
が成り立つ。数値計算においては$T=0$に到達することは出来ないから、1番絶対値が小さい松原周波数は有限となる。それを上式の$i\eta$と見なせば、遅延グリーン関数からトポロジーを評価出来ることが分かる。
こうして遅延グリーン関数から得た有効ハミルトニアン
\begin{eqnarray}
    H_{\mathrm{eff}}(k) = H(k) + \Sigma^{R}(\omega=0,k)
\end{eqnarray}
に対して、本研究では離散化された形式のWilson Loop\cite{asboth2016short,PhysRevB.99.045140,PhysRevB.100.195135}
\begin{eqnarray}
W(k_x) =\langle u(k_x,2\pi)| \displaystyle \prod_{n=1}^{N}P(k_x,2\pi n/N) |u(k_x, 2\pi)\rangle,
\end{eqnarray}
を計算することでトポロジーを判定した。ただし、$|u(k_x,k_y)\rangle$ は$H_{\mathrm{eff}}(k)$の占有された固有状態であり、 $P(\bk)=|u(k_x,k_y)\rangle\langle u(k_x,k_y)|$ は射影演算子である。射影演算子の積は$k_y$が左から右に行くにつれて増加するように並べる。また、数値計算においては$N$ は十分大きく取る必要がある。Wilson Loopの巻き付き数が有限であればその系はトポロジカルに非自明であり、巻き付き数が$0$なら自明であると分かる~\cite{asboth2016short,PhysRevB.99.045140,PhysRevB.100.195135}。

\section{動的平均場理論}%DMFT/NRG
この節では、本研究において自己エネルギーを得る際に用いた手法である動的平均場理論(Dynamical Mean-Field Theory,DMFT)について文献\cite{RevModPhys.68.13}に基づいて説明する。DMFTは端的に言えば自己エネルギーの周波数依存性も取り込める平均場近似であり、イジング模型などで適用される静的な平均場とは異なる。しかし、1つのサイトに着目して自己無撞着方程式を得る点は同じであるので、まずはイジング模型の平均場近似について解説する。

イジング模型のハミルトニアンは
\begin{eqnarray}
    H=-J\displaystyle\sum_{<ij>}J_{ij}S_iS_j -h\sum_i S_i
\end{eqnarray}
で与えられる。ただし、$S_i$はスピンを表し、$J_{ij}$はスピン同士のカップリング、$h$は磁場を表す。また、和記号の$<ij>$は最近接のサイトとだけ和を取ることを意味する。

平均場近似では、特定のサイト($o$とする)に注目し、そのサイトにおける有効的な磁場が掛かったハミルトニアン
\begin{eqnarray}
H_{\mathrm{eff}} = -h_{\mathrm{eff}}S_o
\end{eqnarray}
で系を記述することを考える。他のサイトからの寄与は全て有効磁場
\begin{eqnarray}
    h_{\mathrm{eff}}=h + \sum_{i}J_{oi}m_i=h+zJm
\end{eqnarray}
で表されるとする。ただし、$z$は最近接格子数、$m_i = <S_i>$はサイト$i$の磁化であり、並進対称性$J_{ij} = J,m_i = m$を課した。この$H_\mathrm{eff}$から磁化を計算することにより、自己無撞着方程式
\begin{eqnarray}
    m=\tanh(\beta z + z\beta Jm)
\end{eqnarray}
を得る($\beta$は逆温度)。この方程式を解くことで、イジング模型の相転移を記述できるのであった。

格子上の電子系(例えば本研究で用いた模型)を静的な平均場で扱うと、単に
\begin{eqnarray}
    H_\mathrm{int} = U\displaystyle \sum_i n_{i,a}n_{i,b}\mapsto U\sum_i\Bigl( <n_{i,a}>n_{i,b}+n_{i,a}<n_{i,b}>\Bigr)
\end{eqnarray}
と置き換わるだけであるので、自己エネルギーの周波数依存性を取り込むことができない。したがって、モット転移などを扱えないことになる。その点を改良したのがDMFTであり、DMFTは無限次元で厳密であることが知られている。

DMFTにおいては、格子系を不純物模型にマップする。不純物模型の例としては、例えばアンダーソン模型(single-impurity Anderson Model)
\begin{eqnarray}
    H=\sum_{k\sigma}\epsilon_k a^{\dagger}_{k\sigma}a_{k\sigma} + \sum_{k\sigma}V_k\Bigl(a^{\dagger}_{k\sigma}c_{o\sigma}+h.c\Bigr) + Un_{o\uparrow}n_{o\downarrow}
\end{eqnarray}
がある。本研究ではこの模型を用いた。ただし、$k$は波数、$\sigma$はスピンを表し、$c_{o\sigma},n_{o\uparrow}n_{o\downarrow},$が不純物サイトの電子に対応する演算子、$a_{k\sigma}$はbath中の電子に対応する演算子である。$\epsilon_k$はbathの分散関係を表し、$V_k$はbathと不純物のカップリングを表す。$U$は不純物サイトでの相互作用の強さを表す。DMFTにおける自己無撞着方程式は、不純物模型の不純物サイトにおけるグリーン関数と、元の格子模型における原点のグリーン関数が等しいとすることで得られる。

具体的には、以下の手続きを繰り返すことになる;
(i)格子模型における原点のグリーン関数$\tilde{G}_{0}(\omega)=\sum_{k}G(\omega,k)$を求める。(ii)これを$U=0$での不純物サイトのグリーン関数であるとしてアンダーソン模型を解き、自己エネルギー$\Sigma(\omega)$を得る。(iii)$\Sigma(\omega)$が格子系における原点のグリーン関数の自己エネルギーだとして、格子系の原点グリーン関数$G(\omega)$を計算する。(iv)$G(\omega)$の自由な部分に相当する$\tilde{G}(\omega)=1/\Bigl(G(\omega)^{-1}+\Sigma(\omega)\Bigr)$を計算する。(v)$\tilde{G}(\omega)$を再び$U=0$での不純物サイトのグリーン関数であるとしてアンダーソン模型を解く。

この手続き(iii)においては、相互作用がある不純物模型を解かなければならない。その手法としてはいくつかの方法が提案されているが、本研究においては数値繰り込み群(Numerical Renormalization Group,NRG)を用いた。以下では、NRGについて文献\cite{RevModPhys.80.395}に基づいて簡単に説明する。

NRGは量子論的不純物模型を非摂動的に扱える手法であり、歴史的にはWilsonによって近藤模型に適用され、高温領域と低温領域のクロースオーバー領域での計算が初めてなされた。その後、アンダーソン模型など近藤模型以外の模型にも適用されている。NRGのアイデアは、bathの伝導帯のエネルギー準位を対数的に離散化し、不純物模型を半無限スピン鎖の模型にマップするというものである。以下では、簡単のため伝導帯のエネルギーは区間$[-1,1]$の間にあるとする。

この区間を離散化するために、パラメータ$\Lambda > 1$を導入しする。$\Lambda$はNRG disctretization parameterと呼ばれる。離散化されたエネルギーは$\{\pm\Lambda^n\}_{n=0}^{\infty}$で与えられる。$\Lambda\to 1$の極限で連続的なエネルギースペクトルが回復するが、低エネルギーの様子を調べるのに必要な$n$が大きくなってしまうという問題がある。実際の計算では$\Lambda=2$程度で良い精度が出ることが知られており、本研究においても$\Lambda=2$とした。

具体的な手続きは省略するが、アンダーソン模型はいくつかの近似を用いることで以下のスピン鎖模型に置き換えられる;
\begin{eqnarray}
    H_{\mathrm{spin}}&=H_{\mathrm{imp}} + \sqrt{\frac{\xi_0}{\pi}}\sum_n \Bigl(c^{\dagger}_{o\sigma}b_{o\sigma}+h.c\Bigr) \\
    &+ \sum_{\sigma}\sum_{n=0}^{\infty}\epsilon_n b_{n\sigma}^{\dagger}b_{n\sigma} + t_n\Bigl(b_{n+1\sigma}^{\dagger}b_{n\sigma}+h.c\Bigr)
\end{eqnarray}
ただし、
$\xi_0$は定数で、$\{b_{n\sigma}\}_{n}$は伝導帯電子の生成消滅演算子の線形結合で定義される新たな生成消滅演算子である。

数値計算においては無限のスピン鎖$(n=\infty)$は実装できないため、$n$の最大値は有限である。$n$の最大値$N$を一つ固定する毎にハミルトニアン$H_{\mathrm{spin},N}$が定まるので、ハミルトニアンの列$\{H_{\mathrm{spin},N}\}_N$が定義できる。これを逐次的に対角化するのがNRGの手続きである。$N$を増やすと固有状態の数が増加していってしまうが、これに対しては自然数$N_s$を一つ固定して、常に下から$N_s$個のエネルギーのみを取り出す。本研究においては、$n$の最大値は30とし、$N_s=1500$とした。

\chapter{数値計算結果}
\section{$U=0$の場合}
\begin{figure}
\centering
\includegraphics[width=0.47\columnwidth]{Fig1.pdf}
\includegraphics[width=0.47\columnwidth]{Fig2.pdf}
\caption{(右図)$U=0,M=1.7$での既存の量子計量(四角)とGQM(丸)の波数依存性。(左図)$U=0$での既存の量子計量の積分値(四角)とGQMの積分値(丸)の$M$依存性。}
\label{fig:qmcompare}
\end{figure}
GQMの$U$依存性を調べる前に、$U=0$においてGQMが式~(\ref{existing})で与えられる既存の定義と一致することを確かめる。本研究を通して、GQMの積分では一番小さい$\omega_1$を$\omega_{1\mathrm{min}} = 0.02$とし、波数は区間$[-\pi,\pi]$を100分割して計算した。また、グリーン関数に含まれる微小な$\eta$は$\eta = 0.05$とした。フェルミ分布関数に含まれる温度は$T=0.01$としているが、今回の模型では$T=0$とみなしてよい。 

図$\ref{fig:qmcompare}$がその計算結果である。右図は$M=1.7$での量子計量の波数依存性であり、左図は量子計量の積分値の$M$依存性である。ただし、左図では既存の量子計量の相転移点における発散を防ぐため、式~(\ref{existing})の分母を$(E_n-E_m)^2+\eta^2$で置き換えた。この$\eta$はグリーン関数中の$\eta$と同じ値である。両図をみると、GQMは既存の量子計量とよく一致していることが分かる。完全に一致していない理由としては、$(E_n-E_m)^2+\eta^2$と置き換えていることの他に、以下の理由が考えられる;(i)GQMの計算においては、$\omega_{1\mathrm{min}}$が有限であるので発散を捉えることが出来ない。つまり、相転移点において既存の定義よりも小さくなると考えられる(ii)本来は無限小であるべき$\eta$が有限であることによって、ギャップが少し小さくなる。すると、$\sigma_{\alpha_\beta}(\omega_1,\bk)$が小さい$\omega_1$に対して有限になるので、GQMの式(\ref{defGQM})中の$1/\omega_1$ によって、GQMが大きくなる。(iii)有限温度の効果によって式(\ref{fnfm})の$|f(E_n)-f(E_m)|$が小さくなるので、GQMが少し小さくなる。

このような違いはあるが、両者は以下の共通の振る舞いを示す; 臨界点($M=0,\;2$)において、積分値は発散的になる。この発散は、臨界点においてDCの縦伝導度が有限になることに起因する。また、文献~\cite{PhysRevB.104.045103}において指摘された不等式$\detg \geq \pi |Ch| = \pi$がトポロジカル相において成り立っていることが分かる。さらに、積分値はトポロジカル相において大きく、自明相において小さくなる傾向があることがわかる。(文献~\cite{PhysRevB.104.045103}では様々なチャーン絶縁体において$\detg$が同じ振る舞いを示すことが数値的に確かめられている。)
\section{$U>0$の場合}
次に、$U>0$でのGQMの計算結果を示す。$U$が増加するに伴い、系はトポロジカル相転移を起こすと期待される。これは以下のように理解出来る: 自己エネルギーの効果によってフェルミエネルギー付近のバンド構造は変化し、$a$サイトと$b$サイトのポテンシャル差$M$は$M_{\mathrm{eff}}=M+(\mathrm{Re}\Sigma^R(\omega = 0)_{AA}-\mathrm{Re}\Sigma^R(\omega = 0)_{BB})/2$となる。ただし、$\mathrm{Re}\Sigma^R(\omega = 0)_{AA}$ と $\mathrm{Re}\Sigma^R(\omega = 0)_{BB}$は自己エネルギーの対角成分の実部である。したがって、$M_{\mathrm{eff}}=\{-2,0,2\}$において系はトポロジカル相転移を起こすと予想できる。

実際、$M=1.5$において系はトポロジカル相から自明相へのトポロジカル相転移を起こす。これは図~\ref{fig:M=1.5_Wilson_GQM}に示されているWilson Loopの結果から分かる。Wilson Loopの巻き付き数は$U < U_c \approx 3.5$のとき$1$であり、$U > U_c$では$0$である。

図~\ref{fig:M=1.5_Wilson_GQM}の右図では$M=1.5$におけるGQMの$U$依存性が示されている。この図から、$\int \mathrm{Tr}(g)dk_xdk_y$ と $\int \sqrt{\mathrm{det}(g)}dk_xdk_y$が$U=$の場合と同じ振る舞いを示していると分かる。$\trg$の発散的な振る舞いは相互作用がある系にも現れている。興味深いことに、不等式$\detg \geq \pi$ は相互作用があるトポロジカル相においても成り立っている。さらに、 $\int \mathrm{Tr}(g)dk_xdk_y$ と $\int \sqrt{\mathrm{det}(g)}dk_xdk_y$ はトポロジカル相において大きくなる傾向があると分かる。従って、これらの量は相互作用がある系でもトポロジーを推定するのに役立つ。 
\begin{figure}
\centering
\includegraphics[width=0.47\columnwidth]{Fig3.pdf}
\includegraphics[width=0.47\columnwidth]{Fig4.pdf}
\caption{(右図)$M=1.5$における$W(k_x)$の位相の$k_x$依存性。$U=3.4$(青点)においてはWilson Loopの巻き付き数が$1$であるが、$U=3.5$(赤点)では巻き付き数が$0$である。(左図)$M=1.5$における$\trg$(青線) と $\detg$(赤線)の$U$依存性。相転移点においては、有効的なポテンシャル差(緑線)は$M_{\mathrm{eff}} = M+(\mathrm{Re}\Sigma^R(\omega = 0)_{AA}-\mathrm{Re}\Sigma^R(\omega = 0)_{BB})/2 = 2$となり、自由な系の場合と同じ値であると分かる。化学ポテンシャルは$\mu = U/2$である。}
\label{fig:M=1.5_Wilson_GQM}
\end{figure}

次に、GQMの半正定値性を確かめる。半正定値性を調べるには、$\mathrm{Tr}g(\bk)$ と $\mathrm{det}g(\bk)$見れば良い(図~\ref{fig:trmapM=1.5})。2次元系においては、半正定値性は任意の$\bk$に対して$\mathrm{Tr}g(\bk) \geq 0$ と$\mathrm{det}g(\bk)\geq 0$が成り立つことと同値であることに注意。

図~\ref{fig:trmapM=1.5}を見ると、GQMはBZの$M$点($k_x = k_y = \pi$)の近くで大きくなることが分かる。これは図~\ref{fig:spectral}に示されているスペクトルから理解出来る。エネルギーギャップが$\mathrm{M}$点近くで小さいことが分かる。したがって、$\sigma_{\alpha\beta}(\omega_1,\bk)$は小さい$\omega_1$に対して有限になり、$1/\omega_1$の寄与でGQMが大きくなることが分かる。
\begin{figure}
\centering
\includegraphics[width=\columnwidth]{Fig5.pdf}
\caption{$\mathrm{Tr}g(\bk)$ と $\mathrm{det}g(\bk)$の波数依存性。GQMの半正定値性が成り立っていると分かる。この図では$M=1.5,U=2.2,\mu = U/2$である。}
\label{fig:trmapM=1.5}
\end{figure}
\begin{figure}
\centering
\includegraphics[width=\columnwidth]{Fig6.pdf}
\caption{スペクトル関数のトレース$\mathrm{Tr}A(\omega,\bk)=-\frac{1}{\pi}\text{Tr}\;\text{Im}G^R(\omega,\bk)$を示す。エネルギーギャップが$\mathrm{M}$点で小さくなることが分かる。スペクトルがぼやけているのは自己エネルギーの虚部の影響である。また、系のPHSと反転対称性から$\mathrm{Tr}A(\omega,\bk)= \mathrm{Tr}A(-\omega,\bk)$が成り立っていることが分かる。この図では$M=1.5,U=2.2,\mu = U/2$。}
\label{fig:spectral}
\end{figure}

最後に、$U=2.2$ に固定し、$M$を変えて同じ解析を行う。$M$が増えるに従い、系はトポロジカル相転移を起こす。図\ref{fig:U=2.2_Wilson_GQM}に示されているWilson Loopの結果から、$M\approx 1.69$においてトポロジカル相から自明相への相転移が起きていることが分かる。相互作用の効果によって、相転移が起きる$M$の値は自由な系における値よりも小さくなっている。

それに対応したGQMの$M$依存性も図\ref{fig:U=2.2_Wilson_GQM}の右に示されている。GQMは転移点で再び発散的な振る舞いを示している。さらに、この発散前後の値を比べることで、$M<1.69$ではトポロジカル相だと推測されるが、この結論はWilson Loopの結果と符合する。また、不等式$\detg \geq \pi$はこのパラメータ領域でも成り立っている。


\begin{figure}
\centering
\includegraphics[width=0.47\columnwidth]{Fig7.pdf}
\includegraphics[width=0.47\columnwidth]{Fig8.pdf}
\caption{(左図)$U=2.2$における$W(k_x)$の位相の$k_x$依存性。$M=1.67$(青点)においてはWilson Loopの巻き付き数が$1$であるが、$M=1.69$(赤点)では巻き付き数が$0$である。(左図)式(\ref{defGQM})から計算した$\trg$(青線) と $\detg$(赤線)の$M$依存性。$M_{\mathrm{eff}}=2$のときにトポロジカル相転移が起きている。パラメータは$U=2.2,\mu = U/2$}
\label{fig:U=2.2_Wilson_GQM}
\end{figure}

\chapter{結論}
本研究では、1粒子グリーン関数を用いてBZ上の一般化された量子計量(GQM)を定義し、相互作用がある場合にも量子計量を計算できるようにした。GQMは系の光学伝導度に基づいている。相互作用の効果は1粒子グリーン関数の自己エネルギーを通して取り込まれている。我々は、$T=0$の自由な系においてGQMが既存の量子計量に一致することと、GQMが半正定値であること、つまりGQMが計量として機能することを解析的に示した。

数値計算においては、チャーン絶縁体のtoy model であるQWZ模型に斥力相互作用を加えた系に対して、GQMを計算した。自己エネルギーを得る際はDMFTとNRGを用いた。また、トポロジーを判定する際には有効ハミルトニアンのWilson Loopを用いた。

Wilson Loopによる解析結果から、GQMがトポロジカル相において大きくなり、自明相において小さくなる傾向があることが分かった。さらに、量子体積とチャーン数の間の不等式は相互作用がある場合にも成り立っていることを確かめた。したがって、GQMは自由な系だけでなく相互作用がある系においてもトポロジーとの繋がりがあると分かった。
\chapter*{謝辞}
本研究は、京都大学大学院理学研究科 物理学・宇宙物理学専攻、物理第一教室 凝縮系理論グループにおいて、Robert Peters講師のご指導のもと行われたものです。私自身は心配性な性格から就職活動に時間を割きすぎてしまい、研究の進捗がなかなか生まれない中でも、辛抱強く見守ってくださったPeters講師に感謝します。また、量子計量をグリーン関数で記述するというアイデアを与えてくださったこと、私の下手な英語の校正を丁寧に行ってくれたことにも、心から感謝します。同研究グループの川上則雄教授、柳瀬陽一教授、池田隆介准教授、吉田恒也准教授、手塚真樹助教、大同暁人助教の各先生方にも、凝縮系セミナーを通じて沢山の刺激をもらいました。研究室の先輩方、同期、後輩の皆さんにもよくしていただき、楽しい大学院生活を送ることが出来ました。また、修士1回生の何も分からなかった私に、数値計算や研究テーマについて多くのアドバイスを下さった研究室OBの道下さんに深く感謝します。さらに、学会の参加などで毎回お世話になった秘書の方々、そしてOAとして雇っていただいたiCeMS藤田グループの藤田さん、松田さん、秘書の里中さんにも、御礼を申し上げます。
最後に、就職活動と大学院の研究に追われて精神的に切迫した状態の私を、常に支えてれた家族に感謝します。
\bibliography{shuron.bib}
\end{document}